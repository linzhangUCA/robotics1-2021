\documentclass[12pt,letterpaper]{beamer}
\usetheme{Copenhagen}
\usecolortheme{seahorse}
\setbeamertemplate{section in toc}{\inserttocsection}

\usepackage[utf8]{inputenc}
\usepackage{amsmath}
\usepackage{amsfonts}
\usepackage{amssymb}
\usepackage{graphicx}
\graphicspath{ {./images/} }
\usepackage{multirow}
\usepackage{hyperref}
\hypersetup{
    colorlinks=true,
    linkcolor=blue,
    filecolor=magenta,      
    urlcolor=cyan,
    pdftitle={Overleaf Example},
    pdfpagemode=FullScreen,
}
\usepackage{minted}

\title[Robotics I]
{ENGR 3421: ROBOTICS I}
\subtitle{Python Advanced}

\author[Zhang, Lin]
{Dr. Lin Zhang}
\institute[UCA] % (optional)
{
  Department of Physics and Astronomy\\
  University of Central Arkansas
}
\date[Robotics1 2021] % (optional)
{September 23, 2021}
\logo{\includegraphics[height=1cm]{../images/uca_bear_logo.png}}


%End of title page configuration block
%------------------------------------------------------------

%------------------------------------------------------------
%The next block of commands puts the table of contents at the beginning of each section and highlights the current section:

\AtBeginSection[]
{
  \begin{frame}
    \frametitle{Outline}
    \tableofcontents[currentsection]
  \end{frame}
}
%------------------------------------------------------------

\begin{document}

%The next statement creates the title page.
\frame{\titlepage}

%---------------------------------------------------------
%This block of code is for the table of contents after the title page
\begin{frame}
\frametitle{Outline}
\tableofcontents
\end{frame}
%---------------------------------------------------------

\section{Class}

\begin{frame}{Object-Oriented Programming}

Object Oriented programming (OOP) is a programming paradigm that relies on the concept of classes and objects. 
It is used to structure a software program into simple, reusable pieces of code blueprints (usually called classes), which are used to create individual instances of objects.
\end{frame}

\begin{frame}[fragile]
    \frametitle{Function w/o Return}
        {\scriptsize
        \begin{minted}{python}
def forward(motor1, motor2, speed=1):
    """
    Args:
        motor1: object instantiate from Motor class
        motor2: object instantiate from Motor class
        speed: scalar in range [0,1]
    Return:
        None
    """
    motor1.set_speed(speed)
    motor2.set_speed(speed)
        \end{minted}
        }
\end{frame}

\begin{frame}[fragile]
    \frametitle{Function w/ Return}

        {\scriptsize
        \begin{minted}{python}
def compute_center(ul_coord, ur_coord, lr_coord, ll_coord):
    """
    Args:
        ul_coord: array with shape (2,) or list with length 2
        ur_coord: [x, y]
        lr_coord: e.g. array(321, 456)
        ll_coord: e.g. [321, 456]
    Return:
        center_coord: coordinate of center of the box represented 
        by a list with length of 2.
    """
    mean_x = (ul_coord[0] + ur_coord[0] + lr_coord[0] + ll_coord[0]) / 4
    mean_y = (ul_coord[1] + ur_coord[1] + lr_coord[1] + ll_coord[1]) / 4
    center_coord = [mean_x, mean_y]

    return center_coord
        \end{minted}
        }

\end{frame}

\begin{frame}[fragile]
    \frametitle{Create a Class}

    {\scriptsize
    \begin{minted}{python}
# define a class
class Robot:
    def __init__(self, name, target):
        self.name = name
        self.target = target

# make an instance
bot = Robot(name='bouncer', target=0)
# check status of the instance 
bot.name
bot.target
    \end{minted}
    }
\end{frame}

\begin{frame}[fragile]
    \frametitle{Create a Class}

    {\scriptsize
    \begin{minted}{python}
class Robot:
    def __init__(self, name, target):
        self.name = name
        self.target = target

    def is_marker_detected(self, image):
        flag = False
        (corners, ids, rejects) = cv2.aruco.detectMarkers(image, d, p)
        if self.target in ids:
            flag=True
        return flag
            
    def switch_target(self, new_target):
        self.target = new_target
# use functions 
bot = Robot(name='bouncer', target=0)
im = np.random.uniform(0,255,(640,480,3))
print("Marker detected: {}".format(bot.is_marker_detected(img)))
bot.target
bot.switch_target(4)
bot.target
    \end{minted}
    }
\end{frame}

\begin{frame}[fragile]
    \frametitle{Inherit a Class}

    {\scriptsize
    \begin{minted}{python}
class NewRobot(Robot):

    def switch_target(self, new_target):
        assert new_target < 1000:
        self.target = new_target

    def make_noise(self):
        print("Robot {} made some noise".format(self.name))
# use functions 
bot = NewRobot(name='bouncer', target=0)
im = np.random.uniform(0,255,(640,480,3))
print("Marker detected: {}".format(bot.is_marker_detected(img)))
bot.target
bot.switch_target(1000)
bot.switch_target(999)
bot.make_noise()
    \end{minted}
    }
\end{frame}

\begin{frame}{API and Source Code}
\end{frame}

\begin{frame}{Numpy}
\end{frame}

\end{document}


